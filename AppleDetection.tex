%% ----------------------------------------------------------------
%% AppleDetection.tex
%% ---------------------------------------------------------------- 
\chapter{Related Work in Apple Detection} \label{Chapter:AppleDetection}

In the previous sections it has been discussed that crop monitoring is a manual process and includes laborious tasks such as fruit counting and harvesting. According to \cite{agro_employment}, there is a continuous decrease in agricultural employment. This seasonal decrease in agro-labour coupled with the need for unnecessary expense reduction, make apparent the need for mechanising this process. \cite{baeten2008autonomous} mentions two types of robotic harvesting. The first one is harvesting in \textit{bulk} quantities using machines such as branch shakers, suitable for products like almonds and olives. The second method is suitable for more delicate fruits such as peaches and strawberries which require more gentle handling. Such an autonomous harvesting system consist of two components: \cite{widyartono2019harvesting}) i) the harvesting system and ii) the machine vision system. The machine vision system can be further separated in vision for navigation and vision for fruit detection and localisation (machine vision harvesting). The sensing device that is used most in both cases is a colour CCD camera. Furthermore, deployments can either use monocular cameras or stereo systems. Monocular cameras provide all the advantages a colour camera does, that is colour based segmentation and texture information, but lack of scale estimation if distance from object is not known (\cite{gongal2015sensors}). However, stereo cameras are capable of depth estimation and are less sensitive under various illuminations thus they are preferred (\cite{wang2016localisation}). Such a deployment is more expensive and limits the system from being a real-time detector.



\section{Other Approaches}
\section{Deep Learning Approaches}
\dots