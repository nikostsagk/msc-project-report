%% ----------------------------------------------------------------
%% Thesis.tex
%% ---------------------------------------------------------------- 
\documentclass{ecsthesis}      	% Use the Thesis Style
\graphicspath{{../figures/}}   	% Location of your graphics files
\usepackage{natbib}            	% Use Natbib style for the refs.
\usepackage{xcolor}			% Footnotes on tables
\usepackage{footnote}		% Footnotes on tables
\usepackage{xcolor}
\hypersetup{colorlinks=true}   	% Set to false for black/white printing
\input{Definitions}            		% Include your abbreviations
%% ----------------------------------------------------------------
\begin{document}
\frontmatter
\title      {Improving Apple Detection and Counting Using RetinaNet}
\authors    {\texorpdfstring
             {\href{mailto:n.tsagko@gmail.com}{Nikolaos Chrysovalantis Tsagkopoulos}}
             {Nikolaos Chrysovalantis Tsagkopoulos}
            }
\addresses  {\groupname\\\deptname\\\univname}
\date       {\today}
\subject    {}
\keywords   {}
\maketitle
\begin{abstract}
This work investigates the apple detection problem using the RetinaNet object detection framework and the VGG architecture. After all necessary modifications and hyper-parameter tuning, puts under examination how performance scales with the backbone's network depth. Furthermore, it examines performance through four different proposed deployments for the side-network. An analysis of the relation between performance and training size finds that 10 samples are enough to achieve adequate performance. It also discovers that 200 samples are enough to achieve state-of-the-art performance. Moreover, it proposes a novel lightweight model that achieves an F1-score of 0.908, and inference time of nearly 70FPS outperforming previous state-of-the-art models in both performance and detection rate. Finally, it discusses the results, addresses the model's limitations and weaknesses and provides insights for future work. 
\end{abstract}

\statementoforiginality{
I have acknowledged all sources, and identified any content taken from elsewhere. RetinaNet is based on the open-source implementation provided by Fizyr, licensed under the Apache License 2.0, and is appropriately cited. I did all the work myself and have not been helped by anyone else. The material in the report is genuine, and I have included all the necessary links with my data/code. I have not submitted any part of this work for another assessment. My work did not involve human participants, their cells or data, or animals.
}
\tableofcontents
\listoffigures
\listoftables
%\lstlistoflistings
\listofsymbols{ll}{
$2D$ & two dimensional \\
$AUC$ & area under curve \\
$CCD$ & charge-coupled device \\
$CNN$ & convolutional neural network \\
$CRF$ & conditional random field \\
$FCN$ & fully convolutional network \\
$FN$ & false negative \\
$FP$ & false positive \\
$GMM$ & Gaussian mixture models \\
$GPS$ & global positioning system \\
$GPU$ & graphical processing unit \\
$HSV$ & Hue, Saturation and Value \\
$IoU$ & Intersection over Union \\
$mAP$ & mean average precision \\
$MLP$ & multi-layer perceptron \\
$NIR$ & near infra-red \\
$NMS$ & non-maximum suppression \\
$PCA$ & principal component analysis \\
$ReLU$ & Rectified Linear Unit \\
$RGB$ & Red, Green and Blue \\
$RoI$ & Region of Interest \\
$RPN$ & region proposal network \\
$SGD$ & stochastic gradient descent \\
$SVM$ & support vector machine \\
$TN$ & true negative \\
$TP$ & true positive \\
$WS$ & watershed segmentation \\
}

%\acknowledgements{Thanks to no one.}
%\dedicatory{"If I have seen further it is by standing on the shoulders of Giants." -Isaac Newton}
\mainmatter
%% ----------------------------------------------------------------
%% ----------------------------------------------------------------
%% Introduction.tex
%% ---------------------------------------------------------------- 
\chapter{Introduction} \label{Chapter:Introduction}
The world population is expected to reach 9 billion by the middle of this century. However, food production does not follow this trend, mainly due to climatic change, land restrictions and other factors (\cite{godfray2010food}). According to \cite{mcguire2015fao}, the rate of undernourished population from 1990 to 2014 was declining (784 million people were undernourished in 2015) and since 2014 is on the rise again (820 million) making apparent the need for securing food. Automation in farms can lower the cost of production and help growers maximise fields' potential, making food accessible to everyone.

National Research \cite{NAP5491} defined precision agriculture as a management strategy, which aims to expedite food production by developing decision making and other support systems to help farmers facilitate efficient utilisation of their resources by making use of data gathered with information technologies' techniques. The results of efficient farming, apart from the gains in crop production, can also be measured in terms of invested resources such as labourers, total amounts of water and fertiliser used, thus also reducing the environmental footprint (\cite{zhang2012application}). This thesis aims to investigate the task of apple detection as groundwork to robotic harvesting.

\section{Motivation}
Fruit counting is a manual process done to estimate crop production and create accurate yield mappings. Usually, it is being done in large orchards, under extreme heat and humid conditions, making it a highly intensive and time-consuming labour task. A common farmers' practice to levitate this, is to count the fruits in a small patch of the orchard and then generalise (\cite{bargoti2017fruit}), thus leading in inaccurate results. Other factors such as occlusion by foliage and other fruits or sunlight, contribute to inaccuracy as well.

Furthermore, according to \cite{agro_employment}, there is a continuous decrease in agricultural employment. This seasonal decrease in agro-labour, coupled with the reasons discussed above, further enhance the need for the mechanisation of this process. Therefore, the development of an algorithm that can adequately localise and count fruits in an image is necessary to levitate growers from this task by bringing closer robotic harvesting in agriculture. Besides, accurate detection and counting algorithms can also serve in orchard monitoring and yield mappings by giving continuous measurements about crop production and anticipating pests and diseases.

\section{Problem Formulation and Objectives}
This thesis aims to investigate the accuracy of a deep learning-based object detector framework in detecting apples in images under various illuminations and occlusions. Specifically, this project aims to answer the following research question:

\begin{quote}
\centering 
\textit{"How accurately can RetinaNet detect and count the apples on a given RGB dataset taken by monocular cameras?"}
\end{quote}

The objectives of the proposed research are summarised below:

\begin{itemize}
\item \textbf{Deploying and training RetinaNet on the VGG architecture to improve upon the state-of-the-art results.} VGG architecture (\cite{simonyan2014very}) is an architecture based on blocks of increasing receptive field made by stacking $3\times3$ convolutional filters which preserve feature maps' spatial size. This philosophy enables further exploration in more lightweight models by pruning redundant layers from the backbone, while keeping outputs' size fixed.  
\item \textbf{Identify through systematic experiments the relation between accuracy and the size of the training set.} Deep learning-based methods require large datasets to achieve high performances and generalisation, but such a dataset is in contrast with the motivation of the problem, which is to exempt labourers from doing the labelling task.
\item \textbf{Identify the causes that limit the models' performance.} Apple detection has been tackled with several algorithms on different datasets, thus making comparisons between them impractical. The dataset used in this project was first presented through the work of \cite{bargoti2017deep} and then was also used in the work of \cite{liang2018apple}. Surprisingly, in both cases, the extracted results were quite similar. This observation raises questions about dataset's structure and should be further investigated as a possible factor that limits the models' performance.
\end{itemize}

Deep learning-based object detectors are widely studied in the agricultural sector, especially in the context of fruit detection yielding the state-of-the-art results. Faster - RCNN (\cite{ren2015faster}), has been used repeatedly for apple detection (\cite{sa2016deepfruits}, \cite{bargoti2017deep}, \cite{tao2018rapid}) and other fruits as well, due to its high performance in the general datasets PASCAL-VOC (\cite{everingham2010pascal}) and MS-COCO (\cite{lin2014microsoft}). Recent advances in single-stage detectors, such as RetinaNet (\cite{lin2017focal}) and YOLOv3 (\cite{redmon2018yolov3}), give space for even better results.

\section{Deep Learning in Agriculture}
Image data acquired from cameras can facilitate deep learning techniques, especially convolutional neural networks (CNN) which require spatial 2D inputs (with a depth channel of arbitrary size). \cite{krizhevsky2012imagenet} success of classifying the ImageNet (\cite{deng2009imagenet}) dataset using convolutional neural networks, drew the attention of the agricultural sector. \cite{kamilaris2018deep}, in their recent survey, demonstrated the popularity of deep learning in agriculture by gathering and presenting 40 research papers that make use of it and outperform pre-existing methods. Fruit counting is the second most popular area (after weed identification) and 37 out of these 40 papers were published after 2015. 

\subsection{Advantages of Deep Learning}
The success of CNNs is highlighted by the fact that the need for handcrafted features is getting dismissed. Feature engineering is a time-consuming task and not always evident since it requires manual knowledge and effort. \cite{bargoti2016image} approached apple detection in images using a multi-layer perceptron (MLP) in which images were fed along with some metadata (such as the position of the tree type within the farm and the sun's incidence angle). The gain in performance from metadata was decreasing as the training samples were increasing. 

Furthermore, deep learning models tend to generalise well with large datasets under different types of illumination and occlusions. For example, \cite{rahnemoonfar2017deep} deployed and trained a model entirely with synthetic data, while they tested it on real samples for tomato counting with an accuracy of $91\%$. \cite{chen2017counting} built a fruit counting model trained on challenging datasets with high levels of occlusion, uncontrolled illumination and high similarity between fruit colour and foliage. Both models achieved satisfactory generalisation and were robust in occlusion, light conditions and fruit colour. Training time in Deep Learning models usually takes longer compared to classic approaches (e.g. versus a random forest). However, this is relatively vague since the time a model needs depends on various factors such as resources,  labelling efforts, time spent on optimisation and the depth of the model. A CNN compared with a support vector machine (SVM) is slower to train, but there is a considerable gain in inference time.

\subsection{Disadvantages of Deep Learning}
However, it is worth mentioning some of their drawbacks regarding their training and their results. Very deep models need large datasets to converge, and even acquiring a sufficient dataset size, there should be enough variance among the samples to prevent the model from learning spurious rules and achieve satisfying generalisation. A CNN does not learn anything beyond the expressiveness of the dataset. Data augmentation techniques (artificially increasing the dataset through label preserving transformations) are used in most cases, but there is a generalisation only to a certain extent. Most state-of-the-art deep learning models lack generalising to other datasets making it impossible to compare their results with other methods. Regarding computational resources, CNNs exploit parallel computing maintained by commercial graphic cards (GPU) accelerating their training time on the one hand, but require sufficient GPU memory on the other.

\section{Sensing Modalities}
The most common deployments for orchard monitoring applications use colour CCD cameras, either in monocular setups or in stereo systems. Monocular cameras provide all the advantages a colour camera does, such as colour-based segmentation and texture information, but lack of scale estimation if the distance from the object is not known (\cite{gongal2015sensors}). They are preferred for applications such as fruit detection, fruit counting, and creating structure from motion. However, stereo cameras are capable of depth estimation and are less sensitive under various illuminations; thus, they are preferred for yield mapping estimation and 3D reconstruction (\cite{wang2016localisation}). Finally, CCD cameras, often are accompanied by spectral or thermal cameras to capture features in the non-visible spectrum (\cite{gongal2015sensors}).

\section{Related Work in (Deep) Fruit Detection and Counting}
Before deep learning appearance, previous work in fruit detection required heavy use of feature engineering to encode relevant knowledge that could separate fruits from the background. However, feature engineering depends on expertise from the domain knowledge, which is not an easy task as the same methods do not apply in all fruits. This lack in generalisation led in developing crop-specific systems in which visual attributes such as colour, shape and texture are being exploited differently for every fruit.

Specifically, some examples include the works of \cite{wang2013automated}, who developed a system for yield mapping that could count apples by their visual cues in the HSV colour space from images taken under controlled illumination. \cite{pothen2016texture} built a keypoint detector which utilises intensities and gradient orientations on fruit surfaces under conditions of controlled illumination as well. Also, \cite{hung2015feature} developed a system capable of counting red and green apples by segmenting the image in 5 classes (trunk, ground, sky, leaves, fruit) using a conditional random field (CRF). \cite{gongal2015sensors}, in their review, present numerous examples that apply these techniques. Modern fruit detection and counting approaches are mostly deep learning-based, as the need for feature engineering is not required anymore, and outperform classical methods. Moreover, these techniques tend to work well with RGB data coming from monocular cameras, thus reducing the need for NIR (near infra-red) (\cite{hung2013orchard}) or other data from the non-visible spectrum.

\subsection{Fruit Detection and Counting as a Single Problem}
The dominant approaches for fruit detection and counting are two; the first one faces detection and counting as a single problem using a deep learning object detection framework. This method has been used extensively with Faster R-CNN, through the work of \cite{bargoti2017deep}, \cite{sa2016deepfruits}, \cite{mai2018faster}, \cite{fu2018kiwifruit}, \cite{gene2019multi} and benchmarked under various datasets including mangoes, apples, almonds, sweet peppers and kiwifruits. \cite{liang2018real}, used SSD (\cite{liu2016ssd}) to build a real-time detector for mangoes and later for apples (\cite{liang2018apple}). \cite{tian2019apple} and \cite{koirala2019deep} achieved state-of-the-art performance in mango detection by altering the backbone network of YOLOv3 (\cite{redmon2018yolov3}). This approach detects and counts the fruits very efficiently in one stage, but it is proved insufficient against fruit clusters. 

\subsection{Fruit Detection and Counting as Two Separate Problems}
The second approach, segments the image using a CNN as a blob detector and counts the segmented instances using classical methods such as watershed segmentation (WS) (\cite{beucher1992watershed}), Circular Hough Transformation (CHT) (\cite{atherton1999size}) or Gaussian mixture models (GMM) as seen in \cite{hani2018comparative}, \cite{bargoti2017image} and  \cite{chen2017counting}. It performs better at counting fruits in clusters but is not as accurate as the first method in the detection task. In addition to these recent works, it is worth mentioning the work of \cite{stein2016image}, who introduced a multi-view approach addressing occlusion by detecting and tracking fruits using epipolar geometry and GPS information. Similarly, \cite{liu2018robust}, used an FCN (\cite{long2015fully}) to segment 2D sequential images in order to track the instances across the frames using the Hungarian Algorithm (\cite{kuhn1955hungarian}) with the KLT tracker (\cite{lucas1981iterative}).


%% ----------------------------------------------------------------
%% ObjectDetection.tex
%% ---------------------------------------------------------------- 
\chapter{Object Detection} \label{Chapter:ObjectDetection}

Object detection is the natural extension of the image classification problem. In image classification, for each sample, there is a single output denoting the class of the sample. In mathematical notation for each sample $X$, $\phi(X)$, gives as output an $N-$vector $P_x=(p_1,p_2,...,p_N)$\footnote{$\sum^N_i p_i=1$}, where each element indicates the probability of the sample belonging in the class $n_i \in N$ and $\phi$ denotes the model. Object detection pushes this task a bit further by classifying and localising an instance in the image, instead of simply making a prediction. Therefore, for each sample query, the output includes the predicted class and a set of coordinates that define a bounding box around the object; there could be more than one predictions for a single image. Despite that traditional object detection includes class prediction and a rectangle bounding box around the class, the vision community evolved object detection to image and instance segmentation as well. In image segmentation, the model is called to classify images pixel-wise where each pixel represents a class, while in instance segmentation the model segments the observed classes in objects and background. While object detection was an active field years before deep learning, the most significant and robust algorithms have a deep learning approach. In general, deep learning-based object detectors are divided in \textbf{\textit{double stage detectors}} and \textbf{\textit{single stage detectors}}. An overview of these methods is given in the next sections, along with the most significant models. 

\begin{figure}[!htb]
  \centering
  \subfigure[Classification.]{
    \includegraphics[width=4cm]{figures/ch2/fig1_1.png}
    \label{fig1_1}
  }
  \subfigure[Detection (classification + localisation).]{
    \includegraphics[width=4cm]{figures/ch2/fig1_2.png}
    \label{fig1_2}
  }
  
    \subfigure[Image segmentation.]{
    \includegraphics[width=4cm]{figures/ch2/fig1_3.png}
    \label{fig1_3}
  }
    \subfigure[Instance segmentation.]{
    \includegraphics[width=4cm]{figures/ch2/fig1_4.png}
    \label{fig1_4}
  }
  \caption{Image analysis tasks (Reproduced from \cite{garcia1704review}).}
  \label{fig1}
\end{figure}

\section{Double-Stage Detectors}

The detection process in double stage or \textbf{\textit{region proposal based detectors}}, can be split in two parts. The first part generates class-agnostic regions of interest (RoI) with a bounding box around them and the second part predicts the class of each proposed RoI and refines the bounding box around them. It is worth presenting a popular method in double stage detectors called "R-CNN: Regions with CNN features" (\cite{girshick2014rich}) and its successors Fast R-CNN (\cite{girshick2015fast}) and Faster R-CNN (\cite{ren2015faster}) as Faster R-CNN still stays in fashion due to its high performance.

\subsection{R-CNN}
Before R-CNN was published, object detection was in stagnation without any significant improvement. R-CNN was the first method, that outperformed the previous state-of-the-art CNN method, OverFeat (\cite{sermanet2013overfeat}), increasing accuracy by a notable margin. Specifically, R-CNN achieved a mean average precision (mAP) of 31.4\% on the ILSVRC2013 (\cite{deng2009imagenet})dataset, over OverFeat, which had the previous best result (24.3\%).

R-CNN consists of three parts. The first part generates 2000 class-agnostic regions of interest using selective search (\cite{uijlings2013selective}) with most of them being negative examples. Each proposed region of interest acts as a potential detection defined by its bounding box. Then, these proposals are warped into a fixed size in order to match the architecture's input size criteria. 
The second part is a CNN, pre-trained on the ImageNet dataset (fine-tuned on the new data), and works as a feature extractor from its last fully connected layer.
The 4096-dimensional extracted features from each RoI, are then classified by class-specific SVMs (plus background) in the last step. Next, for each positive prediction, class-specific box regressors regress the bounding box offsets resulting in a refined bounding box.

\begin{figure}[!htb]
  \centering
  \includegraphics[width=12cm]{figures/ch2/fig2.png}
  \caption{R-CNN pipeline as presented in \cite{girshick2014rich}.}
  \label{fig2}
\end{figure}

Although, R-CNN increased performance margin strikingly, has the following notable drawbacks:

\begin{itemize}
  \item The selective search is a heuristic-based algorithm; thus, it is unable to learn anything during the training procedure. 
  \item Training time takes a vast amount of time and space, considering that for each image, 2000 proposals have to be fed to the network. Since training it is a multi-stage pipeline, extracted features have to be cached on disk before passed to the SVMs and box regressors. Notice that since the majority of samples are negatives, the authors adopted hard negative mining forcing the model to focus on hard examples (\cite{felzenszwalb2009object}) in order to accelerate convergence.
  \item It is not suitable for real-time applications as it has a GPU test rate of  47 seconds/images when OverFeat is 9x times faster (\cite{girshick2015fast}). 
\end{itemize}

\subsection{Fast R-CNN}
Fast R-CNN was introduced as the improved version of R-CNN, where the multi-stage network was replaced by a single-stage pipeline able to be trained in one stage.

\begin{figure}[!htb]
  \centering
  \includegraphics[width=12cm, height=3cm]{figures/ch2/fig3.png}
  \caption{Fast R-CNN pipeline as presented in \cite{girshick2015fast}.}
  \label{fig3}
\end{figure}

To achieve this, Fast R-CNN adopted a method from SPPnet (\cite{he2015spatial}) called RoI pooling. R-CNN was handling proposals by warping them into a fixed size and feeding them into the network one by one. On the contrary, SPPnet and Faster R-CNN compute varying sized feature maps of the entire image instead of warped region proposals. Each feature map contains RoI information in a four-tuple $(r,c,h,w)$, where $r, c:$ top-left corner and $w, h:$ width and height. Fixed-size RoI feature vectors are pooled from variable size feature maps through RoI pooling. \fref{fig4} presents how RoI pooling extracts fixed-length feature vectors from feature maps with variable sizes. To extract vectors of fixed-length, different sized proposed RoIs are divided into a fixed number of bins appropriately.

Moreover, the SVMs and the box regressors have been replaced by a joint sibling layer. One layer produces the softmax probability over K + 1 background classes, while the other predicts offsets for the refined bounding boxes. \fref{fig3} illustrates an overview of the model. 

\begin{figure}[!htb]
  \centering
  \subfigure[Feature map with 2 RoIs.]{
    \includegraphics[width=6cm]{figures/ch2/fig4_1.png}
    \label{fig4_1}
  }
  \subfigure[RoIs after RoI pooling.]{
    \includegraphics[width=6cm]{figures/ch2/fig4_2.png}
    \label{fig4_2}
  }
  \caption{RoI pooling layer.}
  \label{fig4}
\end{figure}

By feeding the entire image into the network instead of $\sim$2k proposals one by one, the network avoids redundant computations since many proposals are overlapping each other, hence it is more efficient than its predecessor both in terms of time and memory. Specifically, it is 10x faster than R-CNN in training time and 213x in testing. Furthermore, a joint loss $\mathcal{L}=\mathcal{L}_{cls}+\mathcal{L}_{reg}$ from the softmax and the regression layers optimises the network.

\subsection{Faster R-CNN}
After the replacement of the SVMs with softmax functions, Fast R-CNN was the state-of-the-art object detector, based almost entirely on convolutional neural networks. Its major disadvantage was that the region proposal method, selective search, was not a learning method but a heuristic approach. Soon, this was fixed by the introduction of Faster R-CNN (\cite{ren2015faster}) which entirely dismissed selective search and replaced it with a trainable region proposal network (RPN). RPN is a CNN where the network's output is attached to a class predictor and a box regressor. The class predictor has only two classes (object, not object); thus, the network is trained to propose regions of interest by their "\textit{objectness}". The rest of the network adopts the pipeline of the Fast R-CNN. Faster R-CNN is classified as a double-stage detector even if the region proposal method is not decoupled, anymore, from the main pipeline, due to its 4-step alternate training:
 
\begin{enumerate}
  \item RPN is initialised on the ImageNet weights and gets fine-tuned for the region proposal task.
  \item The detector (Fast R-CNN) is initialised on the ImageNet weights and gets fine-tuned for the detection task, with the proposed regions of interest from the RPN (the two models do not share any convolutional layers yet).
  \item The detector is used to initialise the RPN training, by keeping the weights coming from shared layers constant and updates only the weights coming from the unique layers. At this point, the two models share convolutional layers.
  \item Keeping constant the weights from the common convolutional layers, the detector's unique layers are fine-tuned. 
\end{enumerate}

Faster R-CNN's greatest success is that it consists of a single-stage end-to-end trainable network, entirely based on CNNs with an inference time of 5 FPS.


\section{Single-Stage Detectors}
The downside of double-stage detection is that training time is split into two separate parts, the region proposal and the detection part, hindering real-time detection. Single-stage detectors or \textbf{\textit{regression/classification based detectors}} have the advantage of being really fast. The main reason is that instead of making class predictions and refining bounding boxes on already proposed regions, they apply global rules in every pixel inferring relevant detections without intermediate mechanisms. In this category, the most important frameworks are the SSD (\cite{liu2016ssd}), the YOLO model (\cite{redmon2016you}) and RetinaNet (\cite{lin2017focal}) and will synopsised in the next pages.
 
 \subsection{YOLO}
 YOLO, developed by \cite{redmon2016you} (followed by the improved versions of YOLOv2 \cite{redmon2017yolo9000} and YOLOv3 \cite{redmon2018yolov3}) is a model known for its rapid inference 
 rates ranging between 20-220FPS (depending on the backbone implementation) based entirely on Darknet; a network developed by Redmon himself.
 
The idea behind YOLO is to obtain detections from each pixel individually instead of relying on region proposal methods. An image $W\times H$ is fed into the backbone network, and the output is a feature map of size $S\times S$ coming right after the last convolutional layer. Each pixel in the given feature map represents an area in the original image of $\frac{W}{S} \times \frac{H}{S}$ pixels and is responsible for detecting an object that has its centre on that cell. For example, if an object lies in the bottom right corner of the original image, the pixel $(S,0)$ on the feature map is responsible for its detection. If an object lies in the middle of the image, the pixel $(\frac{S}{2},\frac{S}{2})$ is responsible for its detection. 
 
\begin{figure}[!htb]
  \centering
  \includegraphics[width=12cm]{figures/ch2/fig5.png}
  \caption{A schematic representation of how grid cells detect objects. Reproduced from \cite{redmon2016you}.}
  \label{fig5}
\end{figure} 
 
The actual volume of the feature map sized $S\times S$, is an $S\times S \times (B\times5+C)$ tensor, where 5 in the parentheses implies the four coordinates of the B predictions and an objectness score $p(obj)$ which represents the overlap with the ground truth. Also, for each grid cell, a C-sized vector is predicted, indicating $p(c_i|obj)$. Each cell makes B predictions but only of one class, defined by the probability $p(c_i|obj)$. 

YOLO is limited in making B predictions in each grid cell but only of one class. This limitation makes difficult the detection of small objects in groups. The next versions of YOLO, are capable of making predictions in intermediate layers (from feature maps of different size) enabling detection in various scales and sizes. 
 
\subsection{SSD}
Single Shot Detector (or SSD) was published right after YOLO by \cite{liu2016ssd} and aimed to solve YOLO problems. SSD follows an architecture based on VGG16 (as Faster R-CNN) and operates as YOLO. The first version of Redmon's detector had difficulties detecting objects in different scales due to multiple pooling layers; hence, SSD was aiming to fix this. 

Regarding its architecture, an additional network is attached right after the fifth convolutional block of the VGG16. Classification and regression happen on several feature maps of different size from finer to coarser resolutions. Moreover, grid cells, instead of predicting the conditional class probability $p(c_i|obj)$, predict $p(c_i)$; thus it is necessary the introduction of one more class that acts as a background. While SSD could achieve higher mAP than YOLO with similar detection rates and could detect objects in multiple scales, it was facing difficulties in small object detection, yet this could be relieved by altering the backbone network.  \fref{fig6} shows an overview of the model.
 
\begin{figure}[!htb]
  \centering
  \includegraphics[width=12cm]{figures/ch2/fig6.png}
  \caption{A comparison between SSD and YOLO architectures. Adapted from \cite{liu2016ssd}.}
  \label{fig6}
\end{figure} 
 
The previous sections in this chapter summarised an overview of the most widely used object detectors, the novelties in their architectures, their advantages along with their shortcomings. One the one hand, double-stage detectors produce region proposals and deal with detection in a later stage, and on the other, single-stage detectors divide the images in a grid and produce a bounding box (or not) around a grid cell depending on a probability $p(obj)$. The trade-off between these object detector families is that double-stage detectors usually achieve higher accuracies, while single-stage detectors are much faster. The following sections provide information about technical specifications and concepts met in every object detector in a more detailed manner.

\section{Prior Bounding Boxes (Anchors)}
Anchors act as reference boxes with a predefined size around RoIs, in order to get refined later from the regressor head. Instead of producing new coordinates for every detection, predicting the offsets $(\Delta x, \Delta y, \Delta w, \Delta h)$ of the predefined bounding box is much more efficient. These anchor boxes can be more than one, of different scales and ratios depending on the variance of the dataset (e.g. for pedestrian detection, an elongated anchor box would be more useful than a square one).

Usually, most models define by default 9 anchor boxes with scales of $(2^0, 2^{1/3}, 2^{2/3})$ and ratios of $(1\!:\!2,1\!:\!1,2\!:\!1)$, but this depends on the variation of the objects in the dataset. A common practice to find the most representatives scales and ratios is to apply K-means clustering over the dataset.

\fref{fig7} shows an $8\times8$ feature map where each grid cell tries to fit best nine anchor boxes of area $K\times K$. Each grid cell represents a spatial area of $(\frac{W}{8}, \frac{H}{8})$ in the original image. The cell in $(3,3)$ makes a positive detection though the anchor $A$ if $p(obj)$ is bigger than a certain threshold. The next closest potential prediction's centre would have at least 8 pixels distance from the previous on any adjacent cell; therefore, this feature map's predictions are characterised by a stride = 8. Feature maps of bigger spatial size can achieve more dense predictions. Specifically, stride has a direct relationship with the number of pooling layers, and it is the total number of times the feature map has been downsampled from the original image.

 \begin{figure}[!htb]
  \centering
  \includegraphics[width=7cm]{figures/ch2/fig7.png}
  \caption{A typical topology with 9 anchors acting as prior bounding boxes.}
  \label{fig7}
\end{figure} 

To associate ground truth boxes with suitable anchors, each anchor gets assigned with the overlap between the anchor box and the ground truth box (if any). If the IoU overlap is more than 0.5, then this anchor should be proposed for detection, and it is called a \textbf{\textit{positive anchor}}. If the IoU overlap is less than 0.4, that anchor is going to get suppressed, and it is classified as a \textbf{\textit{negative anchor}}. During training, if an anchor is neither positive nor negative, it does not contribute to the overall loss.

\section{Intersection Over Union}
Intersection over Union (IoU), or Jaccard index, is an evaluation metric used to measure how accurately a predicted box matches or overlaps with the ground truth box. As the name indicates, IoU is the ratio between the intersection and the union of two boxes. In most tasks, as in PASCAL VOC challenge \cite{everingham2010pascal}, if the IoU between the detection and the ground truth is more than 0.5, the detection counts as a true positive.

\begin{figure}[!htb]
  \centering
  \includegraphics[width=3cm]{figures/ch2/fig8.png}
  \caption{The Intersection over Union ratio.}
  \label{fig8}
\end{figure} 

\section{Non-Maximum Suppression}
Usually, objects extend much more in space rather than occupying only a grid cell on the feature map. Thus, many adjacent grid cells, by processing similar context information, may be triggered and produce multiple predictions referring to the same object. A post-processing technique called non-maximum suppression (NMS) deals with this problem by suppressing the detections with IoU exceeding a certain threshold. NMS algorithm can be summarised in the following steps:

\begin{itemize}
  \item In each cell, if $p(obj)>obj_{th}$ the anchor with the highest IoU with the ground truth box is proposed as a prediction. 
  \item The overlapping predictions with IoU greater than the $NMS_{th}$ are suppressed, and the detection with the highest confidence gets proposed as the prediction. 
\end{itemize}

That way NMS suppresses multiple predictions that most likely refer to the same object, but if there are objects of the same class very close to each other, and their ground truth boxes have an overlap higher than the $NMS_{th}$, they will be falsely suppressed but one. 

\section{Multi-task Loss} 
\cite{ren2015faster}, in Faster R-CNN, introduced a coupled loss which combined both the loss from the classification and the regression task. During training, only positive and negative anchors contribute to the loss function; positive anchors in both classification and regression task while negatives only in classification.
The multi-task loss is a weighted combination of the smooth L1 loss and cross-entropy.

\begin{equation}
  L(p_i,t_i) = \frac{1}{N_{cls}}\sum_i{L_{cls}(p_i,p_i^*)}+ \frac{\lambda}{N_{reg}}\sum_i p_i^*{L_{reg}(t_i,t_i^*)}
\end{equation} 

$N_{cls}$, and $N_{reg}$ are normalisation parameters, usually the number of instances in the SGD sample and $\lambda$ a balancing parameter. $p_i^*, p_i$ indicates the ground truth and predicted probability respectively, while $t_i^*, t_i$ is a four-tuple referring to the four ground truth and predicted coordinates that describe the bounding box.

\begin{equation}
    L_{cls}(p,p^*)= \bigg\{
    \begin{array}{ll}
      -log(p) & \text{if } p^*=1 \\
      -log(1-p) &  \text{otherwise}\\
    \end{array}
\end{equation}

Smooth L1 loss is claimed to be more robust to outliers (\cite{ren2015faster}) rather than L2 in which inappropriate learning rates result in exploding gradients. For similar values, or when the manhattan distance between $t_i, t_i^*$ is very small, smooth L1 is much smaller than L1. Additionally, for L1 greater than 1, it can be seen that gradients are constrained to 1. SSD uses the default formula for smooth L1 loss while Faster R-CNN adopts a parameter $\sigma$ which control the point between quadratic and linear loss. Large values of $\sigma$ convert smooth L1 loss to L2 loss.

\begin{equation}
    L_{reg}(t,t^*)= \bigg\{
    \begin{array}{ll}
      0.5\sigma^2(t-t^*)^2 & \text{if } |t-t^*|\leq \frac{1}{\sigma^2} \\
      |t-t^*|-\frac{0.5}{\sigma^2} &  \text{otherwise} \\
    \end{array}
\end{equation}

The following equations show the adopted parameterisation for $t_i$, where $(x^*,y^*,w^*,h^*)$, $(x_a,y_a,w_a,w_h)$ and $(x,y,w,h)$ refer to ground truth, anchor and predicted boxes respectively. 

\begin{align}
t_x		&= (x-x_a)/w_a			&		t_y	&= (y-y_a)/h_a \\
t_w		&= log(w/w_a)			&		t_h 	&= log(h/h_a) \\
t_x^*		&= (x^*-x_a)/w_a		&		t_y^*	&= (y^*-y_a)/h_a \\
t_w^*	&= log(w^*/w_a)		&		t_h^*	&= log(h^*/h_a) 
\end{align}

\section{Focal Loss}
The number of bounding box priors covering the image is usually huge, orders of magnitude greater than the number of instances in the image, creating a class imbalance between negative and positive anchors. To address this class imbalance, \cite{lin2017focal} introduced a weighted cross-entropy loss named "focal loss". Before the focal loss, the most widely adopted method to deal with class imbalance was Online Hard Example Mining explicitly showing the model the hard examples to calculate gradients according to them. Another strategy is feeding the model with a sampling ratio of 1:3 in positives and negatives samples.

Focal loss, down-weights cross-entropy asymmetrically forcing the model to focus on hard examples, that is examples with low confidence $p$. $\gamma$ is the scaling factor, and $\alpha_t$ a balancing factor, the authors state that its precise form is not crucial.

\begin{equation}
    FL(p,p^*)= \bigg\{
    \begin{array}{ll}
      -a_t(1-p)^\gamma log(p) & \text{if } p^*=1 \\
      -a_tp^\gamma log(1-p) &  \text{otherwise}\\
    \end{array}
\end{equation}

\section{RetinaNet}
RetinaNet was introduced as a single-stage adaptation of the Feature Pyramids Networks (FPN, \cite{lin2017feature}) with the enhancement of focal loss. It is the bridge between double-stage and single-stage detection, as it surpassed the performance of Faster R-CNN with detection rates similar to YOLO and SSD.

Detecting objects in different scales is a demanding task, and the standard approach is using image pyramids as a solution. SSD made one of the first attempts in multi-scale object detection by exploiting network's pyramidal feature hierarchy. A convolutional network has the advantage of producing, in each layer, feature maps that get semantically richer with the depth of the layer. This pyramidal hierarchy, results in feature maps, large in resolution but weak in information and more coarse feature maps but semantically rich, obtained from the deeper layers. RetinaNet exploits feature maps from intermediate network layers by combining large resolution semantically weak features with coarser, semantically stronger feature maps. \fref{fig9} shows different types of pyramidal feature exploitation: (a) Building an image pyramid by feeding an image in multiple scales. It used to be a common practice, but it is computationally infeasible. (b) Feeding an image and making predictions from the coarsest, very rich in information, feature map. It is a fast method for making single scale predictions. (c) Detection in different scales by exploiting feature maps in the intermediate layers. It was mainly adopted by the SSD. (d) The FPN method. Detects objects in different scales as the standard SSD detector, but bottom layers are enriched with upsampled coarser, semantically stronger, top layers.

\begin{figure}[!htb]
  \centering
  \includegraphics[width=12cm]{figures/ch2/fig9.png}
  \caption{Different types of pyramidal feature exploitation. Reproduced from \cite{lin2017feature}}
  \label{fig9}
\end{figure} 

\subsection{Architecture}
\fref{fig10} shows how the spatially larger feature maps are enriched by adding upsampled coarser maps. During the bottom-up pathway, the backbone network computes feature maps at several scales with the help of pooling operations. In the case of the VGG architecture, the model consists of 5 convolutional blocks and at the end of each convolutional block, the feature map is downscaled by 2 (the depth of the feature map depends on the architecture) resulting in the hierarchical features $(C_1, C_2, C_3, C_4, C_5)$. For the top-bottom pathway, the feature maps $C_i$ are obtained via lateral connections and convolved with $1\times1$ kernels to produce $C_{i_{reduced}}$ features of the same spatial size but with a fixed depth. The coarser feature map is upsampled by the nearest neighbour method and is added element-wisely with the $C_{i_{reduced}}$ maps. Each merged map finally undergoes a $3\times3$ convolution to reduce the aliasing effect from upsampling. The result is a set of features downscaled by 2, $(P_1,P_2,P_3,P_4,P_5)$, where each $P_i$ contains information from deeper level features. 

\begin{figure}[!htb]
  \centering
  \includegraphics[width=6cm]{figures/ch2/fig10.png}
  \caption{Top-down building block in RetinaNet and FPN. Reproduced from \cite{lin2017feature}}
  \label{fig10}
\end{figure} 

In fact, instead of featurising the output layers in each convolutional block, RetinaNet, attaches at the end of the backbone network two more \textbf{$3\times3$ strided convolutional layers}, continuing downscaling by the same factor the output, without making use of pooling. These outputs are noted as $C_6,C_7$ and the featurised levels are the set $(C_3, C_4, C_5, C_6, C_7)$ which produce the output pyramidal layers $(P_3, P_4, P_5, P_6, P_7)$.

\subsection{Anchor boxes}
RetinaNet adopts the anchor boxes concept to produce RoIs in contrast with the FPN which uses a region proposal network. Due to the number of pooling operations every pyramidal level has undergone, the stride in each $P_i$ layer is $(8, 16, 32, 64, 128)$\footnote{$P_3$ has already been pooled 3 times from block$_1$, block$_2$ and block$_3$, thus it has a stride of $2^3$.}. The effective receptive field of each layer is crucial since it acts as a limiting factor for the size of the anchors and the size of the detected object as a consequence; stride, on the other hand, indicates the density of the predictions. However, in RetinaNet this is not completely true as each $P_i$ is the product of many layer outputs, from various depths with different receptive fields. In general, spatially large $P_i$ features aim in dense small object detection, while coarser $P_i$ maps are responsible for large object detection. Specifically, each layer's anchors have three scales and three ratios. Anchor's base size is $(32^2, 64^2, 128^2, 256^2, 512^2)$ with scales $(2^0, 2^{1/3}, 2^{2/3})$ and ratios of $(1\!:\!2,1\!:\!1,2\!:\!1)$.

\subsection{Prediction subnet}
The subnetwork responsible for detections consist of a \textbf{classification subnet} and a \textbf{box regression subnet}. The classification subnet predicts $p(c_i)$, out of K classes, in each anchor position and is nothing more than 4 stacked $3\times3\times C$ convolutional kernels with ReLU activations, followed by  $3\times3\times KA$ with sigmoid activation. C is the channel depth of $P_i$ and is the same all layers.

The class agnostic box regression subnet follows the same logic with four stacked $3\times3\times C$ convolutional kernels with ReLU activations, followed by $3\times3\times 4A$ with sigmoid activation. The output is a four-tuple for every anchor position with the refined bounding box. The prediction subnet shares its parameters across all pyramidal $P_i$ outputs, following the same philosophy as SSD.

\section{Evaluation Metrics}
Object detection has several evaluation metrics; nevertheless, performance comparisons depend on the problem. For example, in the PASCAL VOC challenge, models compete between those with the highest mAP, while in the apple detection problem, a model with higher F1-score is more accurate. However, the most useful metrics are:

\bigskip
\textbf{Recall}
\bigskip\noindent
\begin{equation}
  \text{Recall} = \frac{TP}{TP+FN}=\frac{TP}{\text{Num. of objects}}
\end{equation} 

Recall is the fraction between successfully detected instances and the total number of instances. A true positive is considered as positive if it has an IoU with the ground truth instance above a certain threshold. Usually, this threshold is set at 0.5.

\bigskip
\textbf{Precision}
\bigskip\noindent
\begin{equation}
  \text{Precision} = \frac{TP}{TP+FP}=\frac{TP}{\text{Predictions}}
\end{equation} 

Precision gives the ability of the model identifying only the relevant instances as it gets lower for every wrong prediction.

\bigskip
\textbf{F1-score}
\bigskip\noindent
\begin{equation}
  \text{F1-score} = 2\times\frac{\text{Precision}\times \text{Recall}}{\text{Precision}+\text{Recall}}\end{equation} 
  
The F1-score is the harmonic mean between precision and recall, giving equal importance in both. It is used, mostly in tasks where the model should have the best ratio between true positives and false positives. Changing the detector's confidence threshold comes with an expense in the rate of true positives along with the rate of false negatives. F1-score indicates the point in the precision-recall curve, where precision and recall have maximum values.

\bigskip
\textbf{Average Precision}
\bigskip\noindent

Average precision (AP) is the area under the precision-recall curve (AUC). The PR curve usually follows a zigzag pattern complicating its calculation. \cite{everingham2010pascal} proposed an alternative way in its calculation by interpolating precision in 11 evenly spaced points. The precision at each recall level takes the maximum value in order to eliminate this zigzag pattern.

\begin{align}
  \text{AP} &= \frac{1}{11}\sum_{r\in\{0,0.1,...,1\}}p_{interp}(r)	&	p_{interp}(r) &= \max_{\tilde{r}:\tilde{r}\geq p(\tilde{r})}
\end{align} 

In this study, the primary evaluation metric is the F1-score, along with other secondary metrics such as the AP, and the mean IoU between predictions and true positives. For the apple counting task, the only metric is the ratio between the number of predictions and the total number of instances in the dataset.
%% ----------------------------------------------------------------
%% Experiment.tex
%% ---------------------------------------------------------------- 
\chapter{RetinaNet Deployment} \label{Chapter:Experiment}
This chapter covers an analytical, detailed review upon the system deployment, the proposed architectures and the hyper-parameter selection accompanied with the respective reasoning behind every choice. It also presents how performance scales in function with the dataset training size. Later proceeds to overall tuning with the most appropriate hyper-parameter values to obtained peak detection and counting.

\section{Dataset Description}
The authors of the papers \cite{bargoti2017image} and \cite{bargoti2017deep} introduced the dataset used in this study and is provided through the Australian Centre or Field Robotics (ACFR), The University of Sydney\footnote{\url{http://data.acfr.usyd.edu.au/ag/treecrops/2016-multifruit/}}. The images were collected using the platform "Shrimp", an unmanned ground vehicle in apple, mango and almond orchards. However, in this study, only the apple subset was used.

Precisely, the dataset consists of random crops from images that span entire trees, trellised in the orchard block. The total number of apple instances in the images represents the total number of fruits in the orchard block. The split in training, validation and test set follows the proposed split from authors' previous work, to provide valid comparisons with the previous literature work. \tref{tab1} summarises an overview of the dataset.

\begin{table}[!htb]
  \centering
  \resizebox{\textwidth}{!}{
  \begin{tabular}{cccccc}
  \toprule
  \textbf{Set} & \textbf{Raw Img. Size} & \textbf{Cropped Img. Size} & \textbf{No. of Img.} & \textbf{Fruit Width} & \textbf{Fruits/Img.} \\
  \midrule
  train 		& $1616\times1232$ & $202\times308$ & 896	& $36.27 \pm 7.55$  & $5.22 \pm 3.37$\\
  val. 		& $1616\times1232$ & $202\times308$ & 112 	& $35.92 \pm 7.83$  & $4.80 \pm 3.10$\\
  test. 		& $1616\times1232$ & $202\times308$ & 112 	& $35.29 \pm 7.30$  & $4.95 \pm 3.21$\\
  train + val. 	& $1616\times1232$ & $202\times308$ & 1008	& $36.23 \pm 7.58$	& $5.17 \pm 3.35$\\
  \bottomrule
  \end{tabular}
  }
  \caption{ACFR dataset description}
  \label{tab1}
\end{table}

The dataset provides circular annotations for the fruits; thus, they were converted to squares with the same width and height. Instead of discarding annotations that exceed the borders of the image, they were clipped to fit the image.
 
\section{Proposed Deployment and Configuration}
\subsection{Hardware and Software specifications}
The deployment was set up on the Keras (\cite{chollet2015keras}) implementation of RetinaNet, provided by Fizyr\footnote{\url{https://github.com/fizyr/keras-retinanet}} and trained using a single NVIDIA Ge Force GTX 1080 Ti provided by the Iridis 5\footnote{\url{https://www.southampton.ac.uk/isolutions/staff/iridis.page}} Computer Cluster. Appendix \ref{pack_versions} includes the complete list with the packages' versions. The repository can be found in \url{https://github.com/nikostsagk/Apple-detection}.

\subsection{Training Details}
The models were trained for 30 epochs of 2000 steps each with $\text{batch size} = 1$. Regarding the optimiser, ADAM (\cite{kingma2014adam}) was used, with an initial learning rate of $10^{-5}$, decreased later by a factor of 10 on epoch 15 and again on 25. It was observed that the models did not show any signs of overfitting as the performance on the validation set was increasing until it stabilised around its maximum value; meanwhile, the validation loss was fluctuating around a minimum value. All models were initialised on the ImageNet VGG16 pre-trained weights unless otherwise stated. 30 epochs of training time took around 80-90 minutes for each model, but the performance had reached its peak from the first 5 epochs.

All models were trained on the training set monitoring performance on the validation set. Performance metrics stated here, refer to performance on the test set unless otherwise indicated.

\subsection{Data Augmentation \& Preprocessing}
Originally, RetinaNet was trained on images with a minimum side of 800. Therefore, first tries adopted this strategy, keeping the ratio unchanged (e.g. $800\times1220)$. Later, to save upon training and inference time, a resolution of $512\times781$ was proposed ensuring that at least one pixel can represent a fruit in the last layer (P7). Finally, the resolution utilised in all models was the original $202\times308$ as no model showed any gain in performance with higher resolutions. By adopting the original resolution, there are considerable earnings in training and inference time giving the opportunity to train multiple models.

The data augmentation techniques used were along with the natural variations of the dataset. Specifically, the augmentations included random flipping along the x-axis with 0.5 chance and random photometric transformations such as \textit{Fancy PCA} (as described in \cite{taylor2017improving}), changes in brightness/contrast, and finally changes in the HSV colour space. Instead of expanding the dataset before training begins, each sample is randomly undergoing a random augmentation during training, avoiding pre-calculations. Besides the augmentation, the ImageNet mean values were subtracted from the dataset as the VGG16 was initially trained that way.

\subsection{Anchor Boxes Configuration}
Instead of using the default anchor boxes' base size, ratios and scales, a common practice used in anchor-based detection pipelines (\cite{redmon2017yolo9000}, \cite{redmon2018yolov3}) is to use k-means clustering, over the dataset where the $k$ centroids' width and height define anchors' base size. Nonetheless, this practice cannot be adopted in this dataset, as the variance among the size of the fruits is small enough, and all centroids would have similar dimensions.

However, the anchor configurations are optimised through a differential evolution search algorithm (\cite{storn1997differential}) as in \cite{zlocha2019improving}. The algorithm starts with the default anchor configuration: base size = $(32, 64, 128, 256, 512)$, ratios = $(1\!:\!2, 1\!:\!1, 2\!:\!1)$ and scales = $(2^{0}, 2^{1/3}, 2^{2/3})$ and through candidate populations try to minimise the distance = $1 - IoU_{avg.}$, that is to maximise the average overlap between anchors and ground truth boxes. For computational efficiency, the algorithm considers only reciprocal ratios = $(1/x, 1, x/1)$ on the validation dataset. \tref{tab2} summarises the proposed anchor configuration. \\

\begin{table}[!htb]
  \centering
  \begin{tabular}{cc}
  \toprule
  \multicolumn{2}{c}{\textbf{Anchor Settings}} \\
  \midrule
base size	& 	$(32^2, 64^2, 128^2, 256^2, 512^2)$ \\
stride 	& 	$(8, \ 16, \ 32, \ 64, \ 128)$ \\
ratios  	&	$(0.805, \ \ 1.0, \ \ 1.242)$ \\
scales  	& 	$(0.696, \ \ 1.0, \ \ 1.313)$ \\
  \bottomrule
  \end{tabular}
  \caption{The suggested anchor configuration proposed by the differential evolution search algorithm. The average IoU between anchor boxes and ground truth, is 0.994.}
  \label{tab2}
\end{table}

 \fref{fig1} illustrates how anchor boxes' area spans the distribution of on the training and validation annotations' area. It is evident that the anchors coming only from the two first layers are enough and span the dataset adequately. A reasonable question would be why the anchor box base size is not reduced even further to span the entire dataset more densely? The reason is that deeper layers, have larger receptive fields and are responsible for the detection of bigger objects; furthermore, the stride increases rapidly in deeper pyramid layers thus smaller anchors would not be able to span the entire feature map densely. This observation justifies the curiosity in exploring shallower alternatives of the original model.
 
\begin{figure}[!htb]
  \centering
  \subfigure[Training dataset.]{
    \includegraphics[width=0.45\textwidth]{figures/ch3/fig1_1.png}
    \label{fig1_1}
  }
  \subfigure[Validation dataset.]{
    \includegraphics[width=0.45\textwidth]{figures/ch3/fig1_2.png}
    \label{fig1_2}
  }
  \caption{The annotation boxes' area distribution, along with the scaled optimised anchor boxes. It can be seen that the anchors coming from the last 3 layers are almost redundant.}
  \label{fig1}
\end{figure}

\subsection{Hyper-parameter Selection}
The IoU threshold to consider a detection as a true positive was set equal to $IoU_{th} = 0.2$ (as in \cite{bargoti2017deep} to perform valid comparisons), unless otherwise stated. The confidence score for the proposed detections was set as $p(c_i) > 0.05$ and the $\text{NMS}_{th}$ was set equal to 0.3 as it found out to be the best after experimentation. To save computation time the maximum proposed detections allowed was set to be 100. 

Lastly, concerning the loss function, tweaking $\alpha$ and $\gamma$ did not yield any difference in performance, so the default parameters $\alpha=0.25$ and $\gamma=2$ were adopted. A notable remark is that the loss function found to be very unstable during training, and the reason was the normalisation parameters, which they take as values the total number of instances in the image. This behaviour was a result of the Fruit/Img. deviation as it can be found in \tref{tab1} very large. To tackle this issue, the normalisation factor was modified, taking values from an exponential moving average of the total instances in the samples.
 
\subsection{Proposed Architectures}

\section{VGG Architectures Comparisons}

\section{Performance - Training Size Relation}

\section{Peak Detection and Evaluation}

\dots
%% ----------------------------------------------------------------
%% Results.tex
%% ---------------------------------------------------------------- 
\chapter{Results And Discussion} \label{Chapter:Results}

\section{VGG Architecture Comparisons}
To estimate the impact of the backbone network and the side-network, each RetinaNet proposed architecture gets deployed on each VGG network separately. \fref{fig1} and \ref{fig2} show how AP and the F1-score scales with the depth of the backbone network.   

\begin{figure}[!htb]
  \centering
  \includegraphics[width=\textwidth]{figures/ch4/fig1.pdf}
  \caption{Comparative results of Avg. Precision versus the four VGG models, for each proposed RetinaNet model. Results were averaged over 10 times.}
  \label{fig1}
\end{figure}

\begin{table}[!htb]
  \centering
  \resizebox{0.75\textwidth}{!}{
  \begin{tabular}{ccccc}
  \toprule
  \textbf{Avg. Precision} & \textbf{VGG11} & \textbf{VGG13} & \textbf{VGG16} & \textbf{VGG19} \\
  \midrule
  Original (202)								&	0.894$\pm$0.004		&	0.925$\pm$0.001		& 	0.929$\pm$0.002		&	0.936$\pm$0.004\\
  $\text{P}_3\text{P}_4\text{P}_5$ (202) 			&	0.891$\pm$0.003		&	0.926$\pm$0.001		& 	0.931$\pm$0.006		&	0.932$\pm$0.004\\
  $\text{P}_\text{i}\text{Multi}$ (202)				&	0.887$\pm$0.005		&	0.925$\pm$0.003 		& 	0.930$\pm$0.005		&	0.932$\pm$0.005\\
  \textbf{$\text{C}_\text{i}\text{Reduced}$} (202) 	&	\textbf{0.907$\pm$0.002}	&	\textbf{0.931$\pm$0.002} 	&	\textbf{0.936$\pm$0.004}	&	\textbf{0.942$\pm$0.001}\\
  \bottomrule
  \end{tabular}
  }
  \caption{Indicative values of Avg. Precision for the selected four VGG models, for each proposed RetinaNet model (\fref{fig1}). Parentheses indicate the input resolution.}
  \label{tab1}
\end{table}

From \fref{fig1}, it can be seen that the average precision scales with the depth of the network steadily. The increment rate is not large enough  to warrant exploring deeper models, as the number of parameters and training time increases adversely. The average precision versus depth diagram, shows that indeed, the depth of the network helps the network detecting more true positives. 

However, the F1-score as shown, in \fref{fig2} does not follow this trend, as it reaches its climax under VGG16 for the $\text{P}_\text{i}\text{Multi}$ and $\text{C}_\text{i}\text{Reduced}$ models, while the Original model exhibits a slight increase.
F1-score, after VGG16, illustrates that neither the depth of the model, nor the side-network deployment can confine the model from predicting false positives.

\begin{figure}[!htb]
  \centering
  \includegraphics[width=\textwidth]{figures/ch4/fig2.pdf}
  \caption{Comparative results of F1-score versus the four VGG models, for each proposed RetinaNet model. Results were averaged over 10 times.}
  \label{fig2}
\end{figure}

\begin{table}[!htb]
  \centering
  \resizebox{0.75\textwidth}{!}{
  \begin{tabular}{ccccc}
  \toprule
  \textbf{F1-score} & \textbf{VGG11} & \textbf{VGG13} & \textbf{VGG16} & \textbf{VGG19} \\
  \midrule
  Original (202)							&	0.832$\pm$0.007		&	0.875$\pm$0.003		& 	0.883$\pm$0.003	&	\textbf{0.887$\pm$0.004}\\
  $\text{P}_3\text{P}_4\text{P}_5$ (202)		&	0.832$\pm$0.005		&	0.872$\pm$0.004		& 	0.885$\pm$0.005	&	0.885$\pm$0.003\\
  $\text{P}_\text{i}\text{Multi}$ (202)			&	0.829$\pm$0.008		&	0.874$\pm$0.002 		& 	0.884$\pm$0.005	&	0.880$\pm$0.003\\
  \textbf{$\text{C}_\text{i}\text{Reduced}$} (202) &	\textbf{0.842$\pm$0.003}	&	\textbf{0.877$\pm$0.004}	&	\textbf{0.886$\pm$0.007}	&	0.882$\pm$0.003\\
  \bottomrule
  \end{tabular}
  }
  \caption{Indicative values of F1-score for the selected four VGG models, for each proposed RetinaNet model (\fref{fig2}). Parentheses indicate the input resolution.}
  \label{tab2}
\end{table}

Concerning the side-network, apart from $\text{C}_\text{i}\text{Reduced}$, the rest RetinaNet deployments, did not demonstrate any striking difference. Under most VGG architectures, $\text{C}_\text{i}\text{Reduced}$ outperformed notably the rest of the architectures, especially in AP. This observation manifests that the problem under study is not affected considerably by the complexity of the side-network in ways such as: integration of high-level semantic feature maps, attaching separate classifier-regressor heads or increasing the depth. Nonetheless, a lighter version of the original RetinaNet, inspired by the SSD pipeline, addresses the apple detection problem through the ACFR dataset satisfactorily.

\tref{tab3} shows that inference time scales with the depth of the model upon increasing the total parameters, apart from the VGG16 and VGG19, which have the same inference time. Lightweight RetinaNet - $\text{C}_\text{i}\text{Reduced}$, holds the highest detection rates among the rest models, as expected.

%$\text{C}_\text{i}\text{Reduced}$ is the lightest model with the smallest memory footprint among the rest; only 19.8M parameters coupled with VGG16, while VGG19 has 20M parameters itself. It achieves high rate detections while maintaining the highest AP = 0.936.

\begin{table}[!htb]
  \centering
  \resizebox{0.8\textwidth}{!}{
  \begin{tabular}{ccccc}
  \toprule
  \textbf{Inference time (FPS)}	  			& \textbf{VGG11} 	& \textbf{VGG13} 	& \textbf{VGG16} 	& \textbf{VGG19} \\
  \midrule
  Original (202)							&	61.0$\pm$3.5		&	58.8$\pm$3.0		& 	55.3$\pm$2.5	&	55.3$\pm$2.5\\
  $\text{P}_3\text{P}_4\text{P}_5$ (202) 		&	71.1$\pm$3.3		&	69.1$\pm$2.5		& 	67.3$\pm$2.0	&	67.3$\pm$2.0\\
  $\text{P}_\text{i}\text{Multi}$ (202)			&	67.8$\pm$4.4		&	67.3$\pm$3.1		& 	65.2$\pm$2.7	&	65.2$\pm$2.7\\
  \textbf{$\text{C}_\text{i}\text{Reduced}$} (202)	& \textbf{74.0$\pm$5.1}	& \textbf{71.1$\pm$4.5}	& \textbf{65.2$\pm$4.3} 	& \textbf{65.2$\pm$4.3}\\
  \bottomrule
  \end{tabular}
  }
  \caption{Inference time for every VGG. Detection rates depend heavily on the No. of parameters of the model and the input resolution. However, the transition from VGG16 to VGG19 did not show any change.}
  \label{tab3}
\end{table}

All VGG models were initialised upon the pre-trained ImageNet weights. Training with random weight initialisation delayed convergence, but did not show any improvement in the results as demonstrated in \cite{bargoti2017deep}. Transfer learning among the VGG models, that is transferring the weights progressively from VGG11 to VGG19 (\cite{simonyan2014very}), did not show any other improvement in performance rather than saving training from a couple of epochs ($\sim3$ minutes with a resolution of $202\times308$).


\section{Performance - Training Size Relation}

This section presents a performance analysis of RetinaNet - $\text{C}_\text{i}\text{Reduced}$ (202)	over the size of the training dataset. For each training subset, N samples were drawn from the training dataset randomly, without replacement (negative samples were discarded). 

 \begin{figure}[!htb]
  \centering
  \subfigure[]{
    \includegraphics[width=0.9\textwidth]{figures/ch4/fig3_1.pdf}
    \label{fig3_1}
  }
  \subfigure[]{
    \includegraphics[width=0.9\textwidth]{figures/ch4/fig3_2.pdf}
    \label{fig3_2}
  }
  \caption{Avg. Precision and F1-score as a function of the number of training images on RetinaNet - $\text{C}_\text{i}\text{Reduced}$ (202), for each VGG network.}
  \label{fig3}
\end{figure}

The subsets were created once and then used for 10 training sessions each. The models were prone to overfitting, particularly those trained on small datasets; thus, early stopping was implemented to obtain the maximum possible performance. \fref{fig3} makes evident, that the model performs satisfactorily already from the first 10 samples. Furthermore, 200 to 500 samples are enough to achieve state-of-the-art performance. From the first 6 samples, RetinaNet outperforms Faster R-CNN in the work of \cite{bargoti2017deep} by 0.1 in every consecutive training subset.



Regarding backbone's network depth, VGG16 and VGG19 show their superiority over the rest architectures consistently. Another interesting feature of \fref{fig3} is that in training sessions consisting more than 10 samples, VGG11 attained the same performance with VGG16-19 only by decupling the dataset.

\section{Peak Detection and Evaluation}
To obtain maximum performance, RetinaNet - $\text{C}_\text{i}\text{Reduced}$ was trained on VGG16, using three resolutions ($202\times308$, $512\times781$ and $800\times1220$) as described in \sref{training_details}, with F1-score being as the primary evaluation metric. Models presented here were trained initially on the training dataset and then both on the training and validation dataset, while being evaluated on the held-out testing dataset.

\begin{savenotes}
\begin{table}[!htb]
  \centering
  \resizebox{0.8\textwidth}{!}{
  \begin{tabular}{ccccc}
  \toprule
  \textbf{Training set and resolution}	& \textbf{AP} 	& \textbf{F1} 	& \textbf{mIoU} 	&  \textbf{$\Delta E$} \\
  \midrule
  train. (202)					&	0.948	&	0.895		& 	0.793	&	0.099\\
  train. (512) 					&	0.953	&	0.903		& 	0.814	&	0.040\\
  train. (800)					&	0.952	&	0.903		& 	0.829	&	\textbf{0.016}\\
  train. + val. (202)				&	0.949	&	0.895		& 	0.807	&	0.081\\
  train. + val. (512)				&	0.954	&	0.905		& 	0.824	&	0.038\\
  train. + val. (800)				&	0.953	&	\textbf{0.907}	& 	0.835	&	0.025\\
  \midrule
  Faster R-CNN (ZF) [1]\footnote{[1]: \cite{bargoti2017deep}}		& 	-		&	0.892		&	-		&	-\\
  Faster R-CNN (VGG16) [1]								& 	-		&	0.904		&	-		&	-\\
  Faster R-CNN (VGG16) [2]\footnote{[2]: \cite{liang2018apple}}	& 	-		&	0.879		&	-		&	-\\
  \midrule
  SSD (300) [2]												& 	-		&	0.883		&	-		&	-\\
  SSD (500) [2]												& 	-		&	0.890		&	-		&	-\\
  \bottomrule
  \end{tabular}
  }
  \caption{Comparing performance of RetinaNet - $\text{C}_\text{i}\text{Reduced}$ (VGG16), trained on the training and on the combined training - validation set, with the state-of-the-art models. Parentheses indicate resolutions. Models were evaluated on $\text{IoU}_{th}=0.2$ and $\text{NMS}_{th}=0.3$.}
  \label{tab4}
\end{table}
\end{savenotes}

\tref{tab4} shows that performance increases with resolution, with the best model achieving a maximum F1-score of \textbf{0.907}, outperforming the previous state-of-the-art model (\cite{bargoti2017deep}). \cite{liang2018apple} gave more emphasis on real-time detection achieving 43.47 FPS using the SSD(300).

RetinaNet - $\text{C}_\text{i}\text{Reduced}$(VGG16), beaten SSD in performance and inference time by using even smaller resolution. A resolution of 800 improved performance, in terms of F1-score by $\sim1\%$, but with a considerable drop in training time versus the smaller resolution of 202. Specifically, models trained on the original image dimensions ($202\times308$), required x4 times less time to train in comparison to the ones that used ($800\times1220$) and also, they had x5.5 times less inference time (12.6FPS vs 65.2 FPS).

Moreover, from \tref{tab4} it can be seen that, while the IoU threshold was set equal to 0.2, as in the previous sections, the mean IoU between ground truths and predictions is quite high, making very accurate localisations.

\begin{figure}[!htb]
  \centering
  \includegraphics[width=0.6\textwidth]{figures/ch4/fig4.pdf}
  \caption{Precision - Recall curves for the RetinaNet - $\text{C}_\text{i}\text{Reduced}$ (VGG16). AP is defined by the area under the curve, while F1-score is the point on the curve where Precision and Recall take maximum values.}
  \label{fig4}
\end{figure} 

\fref{fig4} illustrates the precision-recall curves of the trained models presented in \tref{tab4}. All models maintain very high precision for all recall levels. Precision stays around 1.0 for most recall values and starts to drop after recall takes higher values than 0.8.

\subsection{Performance at Various IoU  and NMS Thresholds}\label{nms_threshold}
In most competitions such as in PASCAL-VOC, a prediction to be considered as positive needs at least an overlap of 0.5 with the ground truth. However, \cite{bargoti2017deep} state that an IoU threshold equal to 0.2 is adequate for accurate fruit detection, eliminating the error cases where the localisation is imperfect. \fref{fig5_1} shows how the precision-recall curve changes under different IoU thresholds. The model continues maintaining the same performance even under more strict settings such as raising the $\text{IoU}_{th}$ from 0.2 to 0.5. Only for $\text{IoU}_{th} \geq 0.8$, performance starts to degenerate.

 \begin{figure}[!htb]
  \centering
  \subfigure[]{
    \includegraphics[width=0.45\textwidth]{figures/ch4/fig5_1.pdf}
    \label{fig5_1}
  }
  \subfigure[]{
    \includegraphics[width=0.45\textwidth]{figures/ch4/fig5_2.pdf}
    \label{fig5_2}
  }
  \caption{(a) Precision - Recall curves for the RetinaNet - $\text{C}_\text{i}\text{Reduced}$ (VGG16) under various $\text{IoU}_{th}$. Model's performance remains supreme even by increasing  $\text{IoU}_{th}$ from 0.2 to 0.5. (b) Performance over $\text{NMS}_{th}$. Performance is not very sensitive for $\text{NMS}_{th}$ between 0.2 and 0.6.}
  \label{fig5}
\end{figure}

\begin{table}[!htb]
  \centering
  \resizebox{0.5\textwidth}{!}{
  \begin{tabular}{cccccc}
  \toprule
  \textbf{$\text{NMS}_{th}$}	& \textbf{0.2} 	& \textbf{0.3}	& \textbf{0.4} 	& \textbf{0.5} 	& \textbf{0.6} 	\\
  \midrule
  \textbf{AP}				&	0.938	&	0.953	&	\textbf{0.955}	&	0.945	&	0.928	\\
  \textbf{F1-score}			&	0.896	&	0.907	&	\textbf{0.908}	&	0.891	&	0.872	\\	
  \bottomrule
  \end{tabular}
  }
  \caption{Indicative values of Avg. Precision and F1-score for selected $\text{NMS}_{th}$ at the default $\text{IoU}_{th}=0.2$. $\text{NMS}_{th}$ of 0.4 pushes performance a bit further.}
  \label{tab5}
\end{table}

\begin{figure}[!ht]
  \centering
  \subfigure[$\text{NMS}_{th}=0.3$]{
    \includegraphics[width=0.38\textwidth]{figures/ch4/fig6_1.png}
    \label{fig6_1}
  }
  \subfigure[$\text{NMS}_{th}=0.5$]{
    \includegraphics[width=0.38\textwidth]{figures/ch4/fig6_2.png}
    \label{fig6_2}
  }
  
  \subfigure[$\text{NMS}_{th}=0.3$]{
    \includegraphics[width=0.38\textwidth]{figures/ch4/fig6_3.png}
    \label{fig6_3}
  }
  \subfigure[$\text{NMS}_{th}=0.5$]{
    \includegraphics[width=0.38\textwidth]{figures/ch4/fig6_4.png}
    \label{fig6_4}
  }
  \caption{$\text{NMS}_{th}=0.3$ performs better against clusters (top-left), while fails in occluded fruits such as in the fruits located in the bottom and the far left of the bottom-left image. $\text{NMS}_{th}=0.5$ succeeds in occluded instances, where lower thresholds fail, while fails separating tight clusters resulting in multiple predictions (top-right).}
  \label{ch4:fig6}
\end{figure}

NMS is responsible for suppressing detections with an overlap over a certain threshold. While it is undoubtedly necessary to avoid multiple detections over the same object, it can be harmful in fruit detection problems. The detector is fundamentally developed to suppress all predictions with $\text{IoU}\geq\text{NMS}_{th}$ in a query, but one. The dataset contains many fruits in tight clusters or occluded by others, thus the right $\text{NMS}_{th}$ is inextricably linked to the dataset, as it depends on its levels of occlusion between its instances. However, the capability of RetinaNet - $\text{C}_\text{i}\text{Reduced}$ (VGG16) to make precise localisations with an mIoU of 0.835 (\tref{tab4}), relaxes the $\text{NMS}_{th}$ bounds to the values in \tref{tab5}. \fref{ch4:fig6} contains sample predictions with different $\text{NMS}_{th}$ values and their impact.

\subsection{Counting}
The dataset does not specify any structure among the trellised trees in the pictures, thus structured yield estimation in the orchard block is impossible. However, $\Delta E$ represents overall count estimates, by using the normalised absolute error over the total number of instances in the dataset. $\Delta E$ seems to take smaller values as F1-score increase, giving an unstructured yield estimate for the whole dataset through the total number of predictions. Instead of tuning the detection confidence threshold to minimise $\Delta E$, it was optimised to take the maximum value, while keeping F1-score unchanged.

%% ----------------------------------------------------------------
%% Conclusion.tex
%% ---------------------------------------------------------------- 
\chapter{Conclusion and Future Work} \label{Chapter: Conclusion}
This thesis studied the apple detection problem, in the context of general fruit detection using, a deep learning-based algorithm, RetinaNet. Previous works have demonstrated repeatedly the superiority of deep learning-based methods for agricultural purposes, especially in the field of fruit detection. The biggest advantage of deep learning is that it entirely dismissed feature engineering, which requires expertise from the domain knowledge; thus it enabled the development of algorithms that apply in universal datasets instead of being crop-specific.

RetinaNet, exploits the pyramidal feature hierarchy of the backbone network to build higher level semantic maps. From each intermediate backbone layer it obtains feature maps of different spatial sizes, each containing different levels of information. Combining coarse feature maps, rich in information, with spatially larger semantically weak maps, results in finer maps, containing large amounts of information. The resulting feature maps, are five pyramidal levels of increasing resolution, each with high content in semantic information, enabling detection in multiple scales.

Despite that the hyper-parameter tuning is essential for efficient optimisation, recent work in fruit detection has not given the necessary attention. In this specific study, the right predefined anchor boxes' sizes, have been configured efficiently, through an evolution search algorithm, enabling easier regression. Furthermore, loss has been modified to take more stable values, accelerating convergence. Moreover, an analysis between the dataset's annotation size distribution and the predictions' sizes from each pyramidal level of RetinaNet, showed that possibly some layers might be redundant advocating a deeper exploration in alternative, simpler architectures, thus, side-network's efficiency was studied through four proposed alternative deployments.

Concerning network's effectiveness, it was demonstrated that performance scales with the depth of the backbone network, however, that was not the case with the side-network. The lightweight RetinaNet - $\text{C}_\text{i}\text{Reduced}$ consistently performed better comparing to more sophisticated architectures, proving that by semantically enriching the feature maps, or increasing side-network's complexity in a different manner, does not help this binary problem.

Afterwards, an investigation between performance and training size relation, showed that even 10 samples are enough for adequate detection. Moreover, maximum performance could be achieved by using only 200-500 samples (at least 2x times less than the original training set) relieving growers from labelling big datasets. This finding raises questions around the dataset, as deep learning algorithms, generally, perform better as the training dataset increases.

In the last section, optimisation for peak detection demonstrated that, indeed, high resolutions yield better results, but with infinitesimal difference. However, increasing resolution, increases training and inference time as well. RetinaNet - $\text{C}_\text{i}\text{Reduced}$(VGG16) reached the maximum performance of AP=\textbf{0.955} and F1=\textbf{0.908} using a resolution of $800\times1220$ outperforming the state-of-the-art (\cite{bargoti2017deep}). The same model trained on dataset's original resolution ($202\times308$), achieved an AP=\textbf{0.948} and an F1-score=\textbf{0.895} with detections rate of \textbf{69.5} FPS, x1.5 times faster than the model of \cite{liang2018apple}. Furthermore, the model managed to maintain its first-rate performance even under the more strict $\text{IoU}_{th}$ of 0.5.

Concerning model's limitations, as shown in \sref{nms_threshold}, the detector suppresses detections with overlap $\text{IoU}\geq\text{NMS}_{th}$, as these predictions are registered as multiple detections (Figures \ref{fig4_2}, \ref{fig4_3} and \ref{ch4:fig6}). However, the severity of this problem depends on the context of the application, the framework has been implemented into; e.g. for robotic harvesting it is not very crucial, as after a fruit has been gathered, the algorithm can update the detected instances through a new image, while for yield estimation applications it is harmful, as the model will always undercount. Furthermore, acquiring data with monocular cameras make the model prone to double counting, as the model does not identify the objects across  frames; thus, farmers should put extra effort obtaining datasets with unique instances. Lastly, monocular RGB cameras are incapable of estimating depth, thus the model struggles distinguishing between foreground and background (\fref{fig3}) resulting in learning spurious rules such as classifying relevant objects by their relative size.

The ACFR dataset consists of crops of images that span entire trees in an orchard block. The proposed technique can be commercialised through a robotic harvesting application from unmanned ground vehicles, as it allows accurate detections from images captured in relatively short distances. As future work, it is left to explore algorithm's full potential in real-time detections, such as equipped on drones flying between orchard rows. This can be studied through sequential images of entire trees without sub-sampling, to investigate if the state-of-the-art performance can be preserved in samples with smaller fruits. Integrating tracking designs on the system such as the Hungarian Algorithm, can provide accurate yield mappings.


 \begin{figure}[!ht]
  \centering
  \subfigure[]{
    \includegraphics[width=0.31\textwidth]{figures/ch5/fig1_1.png}
    \label{fig1_1}
  }
  \subfigure[]{
    \includegraphics[width=0.31\textwidth]{figures/ch5/fig1_2.png}
    \label{fig1_2}
  }
  \subfigure[]{
    \includegraphics[width=0.31\textwidth]{figures/ch5/fig1_3.png}
    \label{fig1_3}
  }
  \caption{Cases of successful predictions: (a) in a tight fruit cluster, (b) between foreground and background fruits (bottom-left) and (c) in a fruit cluster with highly occluded instances. Green boxes represent true positive examples.}
  \label{fig1}
\end{figure}

 \begin{figure}[!ht]
  \centering
  \subfigure[]{
    \includegraphics[width=0.31\textwidth]{figures/ch5/fig2_1.png}
    \label{fig2_1}
  }
  \subfigure[]{
    \includegraphics[width=0.31\textwidth]{figures/ch5/fig2_2.png}
    \label{fig2_2}
  }
  \subfigure[]{
    \includegraphics[width=0.31\textwidth]{figures/ch5/fig2_3.png}
    \label{fig2_3}
  }
  \caption{Error cases where apparently correct predictions are registered as false positives due to lack of annotation. Green and red boxes represent true and false positives respectively.}
  \label{fig2}
\end{figure}

 \begin{figure}[!ht]
  \centering
  \subfigure[]{
    \includegraphics[width=0.31\textwidth]{figures/ch5/fig3_1.png}
    \label{fig3_1}
  }
  \subfigure[]{
    \includegraphics[width=0.31\textwidth]{figures/ch5/fig3_2.png}
    \label{fig3_2}
  }
  \subfigure[]{
    \includegraphics[width=0.31\textwidth]{figures/ch5/fig3_3.png}
    \label{fig3_3}
  }
  \caption{Some hard examples in the training dataset. The model encounters difficulties identifying relevant samples between foreground and background. If the model associates distance with the relative fruit size, this misleading rule indicates overfitting, as it does not apply in general. Green and red boxes represent true and false positives respectively.}
  \label{fig3}
\end{figure}

 \begin{figure}[!ht]
  \centering
  \subfigure[]{
    \includegraphics[width=0.31\textwidth]{figures/ch5/fig4_1.png}
    \label{fig4_1}
  }
  \subfigure[]{
    \includegraphics[width=0.31\textwidth]{figures/ch5/fig4_2.png}
    \label{fig4_2}
  }
  \subfigure[]{
    \includegraphics[width=0.31\textwidth]{figures/ch5/fig4_3.png}
    \label{fig4_3}
  }
  \caption{Interesting false negative predictions from the testing dataset. Errors due to: (a) illumination conditions, (b-c) dubious ground truths. In (b-c), it can be seen that occluded instances with $\text{IoU} \geq \text{NMS}_{th}$ are suppressed to avoid multiple detections of the same object.}
  \label{fig4}
\end{figure}



%\appendix
%%% ----------------------------------------------------------------
%% AppendixA.tex
%% ---------------------------------------------------------------- 
\chapter{Package versions} \label{pack_versions}
The following gets in the way of the text....

\backmatter
\bibliographystyle{ecs}
\bibliography{ECS}
\end{document}
%% ----------------------------------------------------------------
